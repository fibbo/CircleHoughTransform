\documentclass[10pt,twoside]{scrreprt}

\renewcommand{\textfraction}{0.001}
\renewcommand{\topfraction}{0.999}   
\renewcommand{\bottomfraction}{0.999}

\usepackage{graphicx}
\usepackage{amsmath}
\usepackage{amssymb}


\usepackage{a4,color}


\usepackage{caption}
\usepackage{float}


\usepackage[T1]{fontenc}
\usepackage[utf8]{inputenc}



\definecolor{red}{rgb}{1,0,0}
\definecolor{green}{rgb}{0,1,0}
\definecolor{blue}{rgb}{0,0,1}
\definecolor{darkblue}{rgb}{0,0,0.8}

\definecolor{yellow}{rgb}{1,1,0}
\definecolor{lightblue}{rgb}{0,1,1}
\definecolor{magenta}{rgb}{1,0,1}
\definecolor{lightgrey}{rgb}{0.5,0.5,0.5}
\definecolor{grey}{rgb}{0.35,0.35,0.35}
\definecolor{darkgrey}{rgb}{0.2,0.2,0.2}
\definecolor{ockerrot}{rgb}{0.859,0.375,0.152}


% \captionsetup{margin=0pt,font=small,labelfont=sc,labelformat=simple,format=plain,indention=3mm,
%  labelsep=endash,textfont=sf,font=sf,singlelinecheck=false,figurename=Fig.,tablename=Tab.}


\fontfamily{ppl}\selectfont


\begin{document}
\chapter{Introduction}
Test

\chapter{Theory}

\section{RICH Detector} % (fold)
\label{sec:rich_detector}

Particle identification is a fundamental requirement at the LHCb experiment. Meaningful CP-violation measurements are only possible if hadron identification is available hence the ability to distinguish between kaons and pions is  essential.
% section rich_detector (end)


\chapter{Methods}

\section{Histogram methods}

\section{Combinatorial approach}

The combinatorial approach relies on the fact that a circle is uniquely defined by 3 points. With 2 arbotrary points we couldn't tell which side the circle is
going to go. A third point gives us all the information we need. The general idea then is the following:

\begin{enumerate}
\item Build all possible triples of points given the data points
\item For all the point triples calculate the center and the radius of the potential circle
\item Due to constraints in the radius we can drop many of the circles with a radius bigger than a certain threshold
\item Create a histogram with the radius distribution. Peaks in the radius distribution hint to a circle.
\item We scan the radius histogram for peaks and look at the center point histogram for the given radius of a peak. If we have also a peak in the center point histogram
      the set of the points of the triples lie on a circle with a radius and center given by the histogram peaks.
\end{enumerate}

\subsection{Drawback}

There are 2 problems with this method.

\begin{itemize}
\item The combinatorics blow up with a high number of data points \( {n}\choose{3} \)
\item Also assuming we have a infinite amount of data points then we would also find an infinite number of circles
\end{itemize}

\chapter{Results}
\end{document}